% Options for packages loaded elsewhere
\PassOptionsToPackage{unicode}{hyperref}
\PassOptionsToPackage{hyphens}{url}
%
\documentclass[
]{article}
\usepackage{lmodern}
\usepackage{amssymb,amsmath}
\usepackage{ifxetex,ifluatex}
\ifnum 0\ifxetex 1\fi\ifluatex 1\fi=0 % if pdftex
  \usepackage[T1]{fontenc}
  \usepackage[utf8]{inputenc}
  \usepackage{textcomp} % provide euro and other symbols
\else % if luatex or xetex
  \usepackage{unicode-math}
  \defaultfontfeatures{Scale=MatchLowercase}
  \defaultfontfeatures[\rmfamily]{Ligatures=TeX,Scale=1}
\fi
% Use upquote if available, for straight quotes in verbatim environments
\IfFileExists{upquote.sty}{\usepackage{upquote}}{}
\IfFileExists{microtype.sty}{% use microtype if available
  \usepackage[]{microtype}
  \UseMicrotypeSet[protrusion]{basicmath} % disable protrusion for tt fonts
}{}
\makeatletter
\@ifundefined{KOMAClassName}{% if non-KOMA class
  \IfFileExists{parskip.sty}{%
    \usepackage{parskip}
  }{% else
    \setlength{\parindent}{0pt}
    \setlength{\parskip}{6pt plus 2pt minus 1pt}}
}{% if KOMA class
  \KOMAoptions{parskip=half}}
\makeatother
\usepackage{xcolor}
\IfFileExists{xurl.sty}{\usepackage{xurl}}{} % add URL line breaks if available
\IfFileExists{bookmark.sty}{\usepackage{bookmark}}{\usepackage{hyperref}}
\hypersetup{
  pdftitle={Homework 2},
  pdfauthor={amirarsalan faghani 21080056!},
  hidelinks,
  pdfcreator={LaTeX via pandoc}}
\urlstyle{same} % disable monospaced font for URLs
\usepackage[margin=1in]{geometry}
\usepackage{color}
\usepackage{fancyvrb}
\newcommand{\VerbBar}{|}
\newcommand{\VERB}{\Verb[commandchars=\\\{\}]}
\DefineVerbatimEnvironment{Highlighting}{Verbatim}{commandchars=\\\{\}}
% Add ',fontsize=\small' for more characters per line
\usepackage{framed}
\definecolor{shadecolor}{RGB}{248,248,248}
\newenvironment{Shaded}{\begin{snugshade}}{\end{snugshade}}
\newcommand{\AlertTok}[1]{\textcolor[rgb]{0.94,0.16,0.16}{#1}}
\newcommand{\AnnotationTok}[1]{\textcolor[rgb]{0.56,0.35,0.01}{\textbf{\textit{#1}}}}
\newcommand{\AttributeTok}[1]{\textcolor[rgb]{0.77,0.63,0.00}{#1}}
\newcommand{\BaseNTok}[1]{\textcolor[rgb]{0.00,0.00,0.81}{#1}}
\newcommand{\BuiltInTok}[1]{#1}
\newcommand{\CharTok}[1]{\textcolor[rgb]{0.31,0.60,0.02}{#1}}
\newcommand{\CommentTok}[1]{\textcolor[rgb]{0.56,0.35,0.01}{\textit{#1}}}
\newcommand{\CommentVarTok}[1]{\textcolor[rgb]{0.56,0.35,0.01}{\textbf{\textit{#1}}}}
\newcommand{\ConstantTok}[1]{\textcolor[rgb]{0.00,0.00,0.00}{#1}}
\newcommand{\ControlFlowTok}[1]{\textcolor[rgb]{0.13,0.29,0.53}{\textbf{#1}}}
\newcommand{\DataTypeTok}[1]{\textcolor[rgb]{0.13,0.29,0.53}{#1}}
\newcommand{\DecValTok}[1]{\textcolor[rgb]{0.00,0.00,0.81}{#1}}
\newcommand{\DocumentationTok}[1]{\textcolor[rgb]{0.56,0.35,0.01}{\textbf{\textit{#1}}}}
\newcommand{\ErrorTok}[1]{\textcolor[rgb]{0.64,0.00,0.00}{\textbf{#1}}}
\newcommand{\ExtensionTok}[1]{#1}
\newcommand{\FloatTok}[1]{\textcolor[rgb]{0.00,0.00,0.81}{#1}}
\newcommand{\FunctionTok}[1]{\textcolor[rgb]{0.00,0.00,0.00}{#1}}
\newcommand{\ImportTok}[1]{#1}
\newcommand{\InformationTok}[1]{\textcolor[rgb]{0.56,0.35,0.01}{\textbf{\textit{#1}}}}
\newcommand{\KeywordTok}[1]{\textcolor[rgb]{0.13,0.29,0.53}{\textbf{#1}}}
\newcommand{\NormalTok}[1]{#1}
\newcommand{\OperatorTok}[1]{\textcolor[rgb]{0.81,0.36,0.00}{\textbf{#1}}}
\newcommand{\OtherTok}[1]{\textcolor[rgb]{0.56,0.35,0.01}{#1}}
\newcommand{\PreprocessorTok}[1]{\textcolor[rgb]{0.56,0.35,0.01}{\textit{#1}}}
\newcommand{\RegionMarkerTok}[1]{#1}
\newcommand{\SpecialCharTok}[1]{\textcolor[rgb]{0.00,0.00,0.00}{#1}}
\newcommand{\SpecialStringTok}[1]{\textcolor[rgb]{0.31,0.60,0.02}{#1}}
\newcommand{\StringTok}[1]{\textcolor[rgb]{0.31,0.60,0.02}{#1}}
\newcommand{\VariableTok}[1]{\textcolor[rgb]{0.00,0.00,0.00}{#1}}
\newcommand{\VerbatimStringTok}[1]{\textcolor[rgb]{0.31,0.60,0.02}{#1}}
\newcommand{\WarningTok}[1]{\textcolor[rgb]{0.56,0.35,0.01}{\textbf{\textit{#1}}}}
\usepackage{graphicx,grffile}
\makeatletter
\def\maxwidth{\ifdim\Gin@nat@width>\linewidth\linewidth\else\Gin@nat@width\fi}
\def\maxheight{\ifdim\Gin@nat@height>\textheight\textheight\else\Gin@nat@height\fi}
\makeatother
% Scale images if necessary, so that they will not overflow the page
% margins by default, and it is still possible to overwrite the defaults
% using explicit options in \includegraphics[width, height, ...]{}
\setkeys{Gin}{width=\maxwidth,height=\maxheight,keepaspectratio}
% Set default figure placement to htbp
\makeatletter
\def\fps@figure{htbp}
\makeatother
\setlength{\emergencystretch}{3em} % prevent overfull lines
\providecommand{\tightlist}{%
  \setlength{\itemsep}{0pt}\setlength{\parskip}{0pt}}
\setcounter{secnumdepth}{-\maxdimen} % remove section numbering

\title{Homework 2}
\author{amirarsalan faghani 21080056!}
\date{}

\begin{document}
\maketitle

\textbf{Before attempting to solve these homework questions make sure
that you've install \texttt{tinytex} package onto your system with
\texttt{install.packages(tinytex)} and
\texttt{tinytex::install\_tinytex()} commands.}

\vspace{1cm}

\textbf{Question 1} Calculate how many minutes in January.

\begin{Shaded}
\begin{Highlighting}[]
\NormalTok{start_date <-}\StringTok{ }\KeywordTok{as.POSIXct}\NormalTok{(}\StringTok{"2023-01-01 00:00:00"}\NormalTok{)}
\NormalTok{end_date <-}\StringTok{ }\KeywordTok{as.POSIXct}\NormalTok{(}\StringTok{"2023-01-31 23:59:59"}\NormalTok{)}
\NormalTok{total_seconds <-}\StringTok{ }\KeywordTok{difftime}\NormalTok{(end_date, start_date, }\DataTypeTok{units =} \StringTok{"secs"}\NormalTok{)}
\NormalTok{total_minutes <-}\StringTok{ }\KeywordTok{as.numeric}\NormalTok{(total_seconds) }\OperatorTok{/}\StringTok{ }\DecValTok{60}
\KeywordTok{cat}\NormalTok{(}\StringTok{"There are"}\NormalTok{, total_minutes, }\StringTok{"minutes in January 2023."}\NormalTok{)}
\end{Highlighting}
\end{Shaded}

\begin{verbatim}
## There are 44639.98 minutes in January 2023.
\end{verbatim}

\textbf{Question 2} Add the numbers 3 1 4 1 5 9 2 6 without \emph{using
the addition sign}.

\begin{Shaded}
\begin{Highlighting}[]
\KeywordTok{sum}\NormalTok{(}\KeywordTok{c}\NormalTok{(}\DecValTok{3}\NormalTok{, }\DecValTok{1}\NormalTok{, }\DecValTok{4}\NormalTok{, }\DecValTok{1}\NormalTok{, }\DecValTok{5}\NormalTok{, }\DecValTok{9}\NormalTok{, }\DecValTok{2}\NormalTok{, }\DecValTok{6}\NormalTok{))}
\end{Highlighting}
\end{Shaded}

\begin{verbatim}
## [1] 31
\end{verbatim}

\textbf{Question 3} Create a vector named \texttt{x} containing the
series -1, -0.9, \ldots, 0, 0.1, \ldots, 0.9, 1 and print the result.

\begin{Shaded}
\begin{Highlighting}[]
\NormalTok{x <-}\StringTok{ }\KeywordTok{seq}\NormalTok{(}\DataTypeTok{from =} \DecValTok{-1}\NormalTok{, }\DataTypeTok{to =} \DecValTok{1}\NormalTok{, }\DataTypeTok{by =} \FloatTok{0.1}\NormalTok{)}
\KeywordTok{print}\NormalTok{(x)}
\end{Highlighting}
\end{Shaded}

\begin{verbatim}
##  [1] -1.0 -0.9 -0.8 -0.7 -0.6 -0.5 -0.4 -0.3 -0.2 -0.1  0.0  0.1  0.2  0.3  0.4
## [16]  0.5  0.6  0.7  0.8  0.9  1.0
\end{verbatim}

\textbf{Question 4} How do we get R to print the text ``SBF!'' 30 times
without repeatingly typing it?

\begin{Shaded}
\begin{Highlighting}[]
\KeywordTok{cat}\NormalTok{(}\KeywordTok{rep}\NormalTok{(}\StringTok{"SBF! "}\NormalTok{, }\DecValTok{30}\NormalTok{))}
\end{Highlighting}
\end{Shaded}

\begin{verbatim}
## SBF!  SBF!  SBF!  SBF!  SBF!  SBF!  SBF!  SBF!  SBF!  SBF!  SBF!  SBF!  SBF!  SBF!  SBF!  SBF!  SBF!  SBF!  SBF!  SBF!  SBF!  SBF!  SBF!  SBF!  SBF!  SBF!  SBF!  SBF!  SBF!  SBF!
\end{verbatim}

\begin{Shaded}
\begin{Highlighting}[]
\CommentTok{#rep() function }
\end{Highlighting}
\end{Shaded}

\textbf{Question 5} Create two vectors named ``wizards'' and
``ranking''. Let the ``wizards'' include the names Harry, Ron, Fred,
George and Sirius, while the ``ranking'' includes the numbers 4, 2, 5,
1, and 3.

\begin{Shaded}
\begin{Highlighting}[]
\NormalTok{wizards <-}\StringTok{ }\KeywordTok{c}\NormalTok{(}\StringTok{"Harry"}\NormalTok{, }\StringTok{"Ron"}\NormalTok{, }\StringTok{"Fred"}\NormalTok{, }\StringTok{"George"}\NormalTok{, }\StringTok{"Sirius"}\NormalTok{)}
\NormalTok{ranking <-}\StringTok{ }\KeywordTok{c}\NormalTok{(}\DecValTok{4}\NormalTok{, }\DecValTok{2}\NormalTok{, }\DecValTok{5}\NormalTok{, }\DecValTok{1}\NormalTok{, }\DecValTok{3}\NormalTok{)}
\end{Highlighting}
\end{Shaded}

\textbf{Question 6} Print/extract the second element of the wizards
vector.

\begin{Shaded}
\begin{Highlighting}[]
\NormalTok{wizards <-}\StringTok{ }\KeywordTok{c}\NormalTok{(}\StringTok{"Harry"}\NormalTok{, }\StringTok{"Ron"}\NormalTok{, }\StringTok{"Fred"}\NormalTok{, }\StringTok{"George"}\NormalTok{, }\StringTok{"Sirius"}\NormalTok{)}
\KeywordTok{print}\NormalTok{(wizards[}\DecValTok{2}\NormalTok{])}
\end{Highlighting}
\end{Shaded}

\begin{verbatim}
## [1] "Ron"
\end{verbatim}

\textbf{Question 7} Replace the names Fred, George and Sirius in the
vector `wizards' with the names Hermione, Ginny, and Malfoy.

\begin{Shaded}
\begin{Highlighting}[]
\NormalTok{wizards <-}\StringTok{ }\KeywordTok{c}\NormalTok{(}\StringTok{"Harry"}\NormalTok{, }\StringTok{"Ron"}\NormalTok{, }\StringTok{"Fred"}\NormalTok{, }\StringTok{"George"}\NormalTok{, }\StringTok{"Sirius"}\NormalTok{)}
\NormalTok{wizards[}\KeywordTok{c}\NormalTok{(}\DecValTok{3}\NormalTok{, }\DecValTok{4}\NormalTok{, }\DecValTok{5}\NormalTok{)] <-}\StringTok{ }\KeywordTok{c}\NormalTok{(}\StringTok{"Hermione"}\NormalTok{, }\StringTok{"Ginny"}\NormalTok{, }\StringTok{"Malfoy"}\NormalTok{)}
\KeywordTok{print}\NormalTok{(wizards)}
\end{Highlighting}
\end{Shaded}

\begin{verbatim}
## [1] "Harry"    "Ron"      "Hermione" "Ginny"    "Malfoy"
\end{verbatim}

\textbf{Question 8} Anyone who hasn't read Harry Potter (like the
professor of this class) needs tags to know who these characters are.
Name the elements of the \texttt{wizards} vector as \textbf{Lead},
\textbf{Friend}, \textbf{Friend}, \textbf{Wife} and \textbf{Rival}.
Print the results.

\begin{Shaded}
\begin{Highlighting}[]
\NormalTok{wizards <-}\StringTok{ }\KeywordTok{c}\NormalTok{(}\StringTok{"Harry"}\NormalTok{, }\StringTok{"Ron"}\NormalTok{, }\StringTok{"Fred"}\NormalTok{, }\StringTok{"George"}\NormalTok{, }\StringTok{"Sirius"}\NormalTok{)}
\KeywordTok{names}\NormalTok{(wizards) <-}\StringTok{ }\KeywordTok{c}\NormalTok{(}\StringTok{"Lead"}\NormalTok{, }\StringTok{"Friend"}\NormalTok{, }\StringTok{"Friend"}\NormalTok{, }\StringTok{"Wife"}\NormalTok{, }\StringTok{"Rival"}\NormalTok{)}
\KeywordTok{print}\NormalTok{(wizards)}
\end{Highlighting}
\end{Shaded}

\begin{verbatim}
##     Lead   Friend   Friend     Wife    Rival 
##  "Harry"    "Ron"   "Fred" "George" "Sirius"
\end{verbatim}

\textbf{Question 9} 26 students entered the PEC206 midterm exam. The
grades of these students are: 18, 95, 76, 90, 84, 83, 80, 79, 63, 76,
55, 78, 90, 81, 88, 89, 92, 73, 83, 72, 85, 66, 77, 82, 99 and 87. Save
test scores in a vector named `scores'. Calculate the mean, median, and
range of exam grades.

\begin{Shaded}
\begin{Highlighting}[]
\NormalTok{scores <-}\StringTok{ }\KeywordTok{c}\NormalTok{(}\DecValTok{18}\NormalTok{, }\DecValTok{95}\NormalTok{, }\DecValTok{76}\NormalTok{, }\DecValTok{90}\NormalTok{, }\DecValTok{84}\NormalTok{, }\DecValTok{83}\NormalTok{, }\DecValTok{80}\NormalTok{, }\DecValTok{79}\NormalTok{, }\DecValTok{63}\NormalTok{, }\DecValTok{76}\NormalTok{, }\DecValTok{55}\NormalTok{, }\DecValTok{78}\NormalTok{, }\DecValTok{90}\NormalTok{, }\DecValTok{81}\NormalTok{, }\DecValTok{88}\NormalTok{, }\DecValTok{89}\NormalTok{, }\DecValTok{92}\NormalTok{, }\DecValTok{73}\NormalTok{, }\DecValTok{83}\NormalTok{, }\DecValTok{72}\NormalTok{, }\DecValTok{85}\NormalTok{, }\DecValTok{66}\NormalTok{, }\DecValTok{77}\NormalTok{, }\DecValTok{82}\NormalTok{, }\DecValTok{99}\NormalTok{, }\DecValTok{87}\NormalTok{)}

\NormalTok{mean_score <-}\StringTok{ }\KeywordTok{mean}\NormalTok{(scores)}
\KeywordTok{print}\NormalTok{(}\KeywordTok{paste}\NormalTok{(}\StringTok{"Mean score:"}\NormalTok{, mean_score))}
\end{Highlighting}
\end{Shaded}

\begin{verbatim}
## [1] "Mean score: 78.5"
\end{verbatim}

\begin{Shaded}
\begin{Highlighting}[]
\NormalTok{median_score <-}\StringTok{ }\KeywordTok{median}\NormalTok{(scores)}
\KeywordTok{print}\NormalTok{(}\KeywordTok{paste}\NormalTok{(}\StringTok{"Median score:"}\NormalTok{, median_score))}
\end{Highlighting}
\end{Shaded}

\begin{verbatim}
## [1] "Median score: 81.5"
\end{verbatim}

\begin{Shaded}
\begin{Highlighting}[]
\NormalTok{range_score <-}\StringTok{ }\KeywordTok{range}\NormalTok{(scores)}
\KeywordTok{print}\NormalTok{(}\KeywordTok{paste}\NormalTok{(}\StringTok{"Range of scores:"}\NormalTok{, range_score[}\DecValTok{2}\NormalTok{] }\OperatorTok{-}\StringTok{ }\NormalTok{range_score[}\DecValTok{1}\NormalTok{]))}
\end{Highlighting}
\end{Shaded}

\begin{verbatim}
## [1] "Range of scores: 81"
\end{verbatim}

\textbf{Question 10} In 2015, Nilay had an annual income of 22,000 TL,
and total expenses of 3,000 TL. In 2016, his annual income was 67,000
TL, and his total expenses were 23,000 TL. In 2017, his annual income
was 70,000TL, and his total expenses were 32,000TL. Finally, in 2018,
his annual income was 72,000 TL and his total expenses were 35,000 TL.
To save this information, create 3 different vectors named `years',
`income' and `expenses'. Calculate Nilay's annual savings and save these
values in a vector named `savings'.

\begin{Shaded}
\begin{Highlighting}[]
\NormalTok{years <-}\StringTok{ }\KeywordTok{c}\NormalTok{(}\DecValTok{2015}\NormalTok{, }\DecValTok{2016}\NormalTok{, }\DecValTok{2017}\NormalTok{, }\DecValTok{2018}\NormalTok{)}
\NormalTok{income <-}\StringTok{ }\KeywordTok{c}\NormalTok{(}\DecValTok{22000}\NormalTok{, }\DecValTok{67000}\NormalTok{, }\DecValTok{70000}\NormalTok{, }\DecValTok{72000}\NormalTok{)}
\NormalTok{expenses <-}\StringTok{ }\KeywordTok{c}\NormalTok{(}\DecValTok{3000}\NormalTok{, }\DecValTok{23000}\NormalTok{, }\DecValTok{32000}\NormalTok{, }\DecValTok{35000}\NormalTok{)}
\NormalTok{savings <-}\StringTok{ }\NormalTok{income }\OperatorTok{-}\StringTok{ }\NormalTok{expenses}
\KeywordTok{print}\NormalTok{(}\KeywordTok{paste}\NormalTok{(}\StringTok{"Years:"}\NormalTok{, years))}
\end{Highlighting}
\end{Shaded}

\begin{verbatim}
## [1] "Years: 2015" "Years: 2016" "Years: 2017" "Years: 2018"
\end{verbatim}

\begin{Shaded}
\begin{Highlighting}[]
\KeywordTok{print}\NormalTok{(}\KeywordTok{paste}\NormalTok{(}\StringTok{"Income:"}\NormalTok{, income))}
\end{Highlighting}
\end{Shaded}

\begin{verbatim}
## [1] "Income: 22000" "Income: 67000" "Income: 70000" "Income: 72000"
\end{verbatim}

\begin{Shaded}
\begin{Highlighting}[]
\KeywordTok{print}\NormalTok{(}\KeywordTok{paste}\NormalTok{(}\StringTok{"Expenses:"}\NormalTok{, expenses))}
\end{Highlighting}
\end{Shaded}

\begin{verbatim}
## [1] "Expenses: 3000"  "Expenses: 23000" "Expenses: 32000" "Expenses: 35000"
\end{verbatim}

\begin{Shaded}
\begin{Highlighting}[]
\KeywordTok{print}\NormalTok{(}\KeywordTok{paste}\NormalTok{(}\StringTok{"Savings:"}\NormalTok{, savings))}
\end{Highlighting}
\end{Shaded}

\begin{verbatim}
## [1] "Savings: 19000" "Savings: 44000" "Savings: 38000" "Savings: 37000"
\end{verbatim}

\end{document}
